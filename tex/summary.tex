%# -*- coding: utf-8-unix -*-
%%==================================================
%% conclusion.tex for SJTUThesis
%% Encoding: UTF-8
%%==================================================

\begin{summary}

本次毕设课题的工作成功地完成了Acemap的2.0版搜索平台的部署。2.0版本的查询系统成功解决了在1.0版本中存在的查询速度与索引速度严重不足和功能缺乏的问题。对于查询速度不足的问题,我采用了将索引文件映射到内存的方法,成功将查询的速度相对原查询系统相比提升了三倍以上;对于索引速度不足的问题,我采用了文件导入法代替数据库导入法,将索引的平均速度提高了超过一百倍,实现了完整索引的成功建立。在另一方面,我实现了查询系统的分布式部署,成功在两台服务器上搭建了系统的分布式云服务,进一步强化了搜索引擎的搜索性能与稳定性。搜索平台的部署工作到此便告一段落,基本的平台维护方式我都整理到了论文的附录中,可以参照附录进行系统的后期维护。

在知识图方面,我一开始的想法是作出一个强理论性的系统,同时也要兼顾系统的展示效果。但是在实际操作中,不同种类的图的展示效果千差万别,并没有一个固定的理论去说按什么标准组织的图的展示效果是最好的,所以后来这部分的工作中,展示效果的优化成了工作的重点,理论部分的工作反而相对随性,只是根据一些参考论文的思想构建了一套自己的规模控制算法,并在自己的系统中使用。由于最终的目标是完成知识图的动态生成,根据用户的输入查询语句直接反馈出相应的知识图,所以算法的复杂度的限制相当高,很多迭代算法因为收敛速度不够也因之被弃用。在论文使用的算法中,使用了一次递归的规模值计算算法和收敛速度较快的Kmeans算法,算法复杂度并不高,在实际应用中也有着较好的效果。

很高兴本次毕设课题的任务书中的都能够被圆满地解决,在解决一个个实际问题的过程中,我对搜索引擎这一应用有了更全面的认识,也学到了各种算法与软件方面的知识。感谢这次毕业设计的机会,让我有了许多收获,也激励我在今后研究生阶段的学习和工作中虚心学习,勇于面对一切困难与挑战。
\end{summary}
