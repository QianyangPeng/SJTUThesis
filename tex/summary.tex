%# -*- coding: utf-8-unix -*-
%%==================================================
%% conclusion.tex for SJTUThesis
%% Encoding: UTF-8
%%==================================================

\begin{summary}

本次毕设课题的工作成功地完成了Acemap的2.0版搜索平台的部署。2.0版本的查询系统成功解决了在1.0版本中存在的索引速度严重不足和功能缺乏的问题,并完成了全部超过1.2亿篇论文的导入工作和诸多附加功能的实现。在另一方面,查询系统实现了分布式部署,在两台服务器上搭建了一个Solr云服务器,进一步强化了搜索引擎的搜索性能与稳定性。搜索平台的部署功能到此便告一段落,但是目前问题的解决并没有十全十美。对于目前超过1.2亿篇论文的查询系统,无论是对于单台系统还是分布式系统,目前一次查询的花费时间约在300ms-3000ms之间,这个时间花费实际上是偏长的。究其原因,是机械硬盘的I/O速度限制,导致大部分时间被花费在了读写文档上。新浪网的查询系统对这个问题有过很好的解决方案,他们同样是采用了Solr架构的搜索引擎,但是用内存映射硬盘的技术将索引文件直接存放在内存中,通过读取内存访问索引,从而将查询速度加快了1000倍左右。对我们的系统来说,索引文件的总大小约为61GB,分布式系统中索引文件总大小大于120GB。然而,两台服务器的总内存也仅有240GB,同时还要支持网站上Redis,MySQL等大量其他应用的使用,实际上是并不足以支持我去尝试内存映射硬盘保存索引的。因此,这种美好的想法也只能暂时作罢。也许,等到我们的学术网站能有一个独立的服务器作为搜索引擎服务器的时候,可以再进行此番改良,相信到时候查询的速度会有一个巨大的提升。2.0版本成功将索引建立速度提高了100倍,在未来可能有的3.0版本中,要将查询速度也提高100倍,这是我对我们研究小组的未来工作的希冀。

在知识图方面,我一开始的想法是作出一个强理论性的系统,同时也要兼顾系统的展示效果。但是在实际操作中,不同种类的图的展示效果千差万别,并没有一个固定的理论去说按什么标准组织的图的展示效果是最好的,所以后来这部分的工作,展示效果的优化成了主要工作,理论部分的工作反而相对随性,只是根据一些参考论文的思想构建了一套自己的规模控制算法,并在自己的系统中使用。由于最终的目标是完成知识图的动态生成,根据用户的输入查询语句直接反馈出相应的知识图,所以算法的复杂度的限制相当高,很多迭代算法因为收敛速度不够也因之被弃用。

我毕业后将会去伊利诺伊大学厄巴纳-香槟分校攻读\emph{编程语言、形式化方法与软件工程}方向的学硕学位,因此,很高兴能在上海交通大学的毕业设计中能选择和自己研究生方向有着紧密联系的课题,并顺利完成任务书下达的任务。这次毕业设计代表着我本科学习的结束,也代表着我专业学习道路的开始,希望自己能够再接再厉,去迎接新的变化与挑战。

\end{summary}
