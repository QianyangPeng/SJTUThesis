%# -*- coding: utf-8-unix -*-
%%==================================================
%% abstract.tex for SJTU Master Thesis
%%==================================================

\begin{abstract}

\large 本毕业论文的题目为学术搜索引擎大规模查询系统的建立与优化。学术搜索引擎是一种针对学术资源进行搜索的搜索引擎,它可以通过论文标题,论文作者,会议期刊等关键词对论文或其它学术资源进行检索。针对这一课题,本论文利用已有的拥有超过1.2亿篇论文的学术数据库,从两个方面研究了课题的相关问题。在一方面,本论文讨论了如何为学术搜索网站建立一个高效稳定的大规模查询系统。首先,论文从系统架构与索引结构上出发,对查询系统完成了初步的设计与实现,并通过索引文件映射到内存的方法,实现了未预热系统上查询速度的飞跃式的提高;在此基础上,论文通过提出文件导入法这一全新的索引建立手段,成功地大大加速了文档的索引速度,解决了之前困扰已久的索引建立速度严重不足的问题;最后,论文讨论了如何对查询系统进行功能扩展与分布式部署,不仅为学术搜索网站添加了关键词高亮与结果统计的功能,而且用两台计算机完成了分布式云搜索服务的搭建。在另一方面,论文基于查询系统研究了根据搜索条件动态生成知识层级图的相关算法,从实现角度进行了知识层级图的网页端可视化,并从理论角度给出了知识层级图的规模控制算法,为查询系统添加了一个新颖而实用的扩展功能。

\keywords{\large 搜索引擎 \quad 数据导入 \quad 分布式系统 \quad 知识图谱}
\end{abstract}

\begin{englishabstract}

The title of this thesis is The Implementation and Optimization of Large-Scaled Academic Searching Platform. An academic search engine is a kind of search engine that aims at indexing and searching academic resources, and it can search papers and other relevant academic resources using their titles, authors, venues and keywords. Here I studied this thesis using a academic paper database with over 120 million papers from two aspects. On the one hand, this paper discussed how to build a high-performance and robust large-scaled searching platform. Firstly, we described how to implement a basic searching platform from the prospective of system architecture and index structure and projected indexed file to memory thus achieving a huge leap in the query performance. Secondly, the paper put forward an innovative data import method -- file import method for academic papers, which had greated accelerated the index building speed and solved a long-standing problem. Lastly, the paper discussed about the function expansion and how to make a distributed deployment to the platform. On the other hand, this paper studied the relevant algorithm of generating dynamic knowledge graph based on searching platform and searching keywords, and implemented a user interface of the knowledge graph, which acted as a new function for our searching platform.

\englishkeywords{\large search engine, data import, distributed system, knowledge graph}
\end{englishabstract}

