%# -*- coding: utf-8-unix -*-
%%==================================================
%% chapter01.tex for SJTU Master Thesis
%%==================================================

%\bibliographystyle{sjtu2}%[此处用于每章都生产参考文献]
\chapter{学术搜索引擎查询系统架构与索引结构的设计}
\label{chap:c2}
\section{工作平台}
    查询系统采用的软硬件工作平台如下:
    \subsection{硬件部分}
        本论文讨论的查询系统的构建方式分为两个阶段。论文的第2-4章中着重讨论了在单台服务器上独立部署服务器的相关内容。在第五章中讨论了包含两台服务器的分布式集群的建立方法。
        
        在单机版本中使用的服务器硬件配置如下:
        \begin{itemize}
        \item CPU: 64位 Intel(R) Xeon(R) CPU E5-2630 v3 @ 2.40GHz * 32
        \item 内存大小:约128GB
        \item 硬盘空间:约800TB
        \end{itemize}
        在分布式服务器中,除了用到了上述服务器之外,还使用了一台备用服务器用以建立集群。该服务器的硬件配置如下:
        \begin{itemize}
        \item CPU: 64位 Intel(R) Xeon(R) CPU E5-2650 v3 @ 2.30GHz * 40
        \item 内存大小:约128GB
        \item 硬盘空间:约4TB
        \end{itemize}
        论文中所有功能的实现与验证都基于该硬件配置。由于查询系统索引规模较大,在单台服务器中索引总大小约为60G,分布式服务器中总大小约为120G,因此将该系统移植到配置较低的计算机时可能会出现问题。
    \subsection{软件部分}
        服务器的操作系统为Ubuntu 14.04.1,Linux内核版本号为3.19.0-25,本论文实验部分对操作系统版本无严格要求。同时,论文采用了Solr6.5.0作为搜索服务器的实现方案。Solr是一个基于Lucene编写的开源搜索平台,由于查询系统建立与优化的重点并不在从倒排索引开始的底层实现上,因此论文全部基于该平台进行系统的实现与部署。

        在分布式集群中,查询系统采用了Zookeeper作为多个Solr内核的管理平台,该平台可以自动管理内核的协调工作,包括配置文件统一管理,任务队列,计算资源分配,主服务器选举,宕机处理等。
    
\section{系统架构}
    查询系统分为两个部分,一部分在网站后台PHP代码中实现,主要控制用户输入的处理与分析,并将分析后的结果传给搜索服务器后台;一部分在Solr平台上实现,主要负责文档的处理,索引的建立与请求的标记解析。这一阶段我们暂时只考虑单台服务器架构下的系统架构。单台服务器上的系统架构见表(\ref{fig:sys_archi})。

\begin{figure}[!htp]
    \centering
    \resizebox{9cm}{!}{\begin{tikzpicture}[node distance=2cm]
    \node (view) [process] {网站视图};
    \node (controller) [process, right of=view, xshift=2cm] {网站控制};
    \node (qprocess) [process, below of=controller] {查询预处理};
    \node (qtokenizer) [process, below of=qprocess] {查询解析};
    \node (database) [process, right of=controller, xshift=3cm] {数据库};
    \node (handler) [process, below of=database] {文档导入处理};
    \node (dtokenizer) [process, below of=handler] {文档分析};
    \node (index) [process, below of=qtokenizer, xshift=2.5cm] {索引服务器};

    \draw [arrow](view) -- (controller);
    \draw [arrow](controller) -- (view);
    \draw [arrow](controller) -- (qprocess);
    \draw [arrow](qprocess) -- (qtokenizer);
    \draw [arrow](qtokenizer) -- (index);
    \draw [arrow](database) -- (handler);
    \draw [arrow](handler) -- (dtokenizer);
    \draw [arrow](dtokenizer) -- (index);
\end{tikzpicture}
}
    \bicaption[fig:sys_archi]{单服务器系统架构}{单服务器系统架构}{Fig}{System Architecture (Singled)}
\end{figure}

    该架构各部分的作用如下:
    \begin{itemize}
    \item 网站视图:负责网页的显示效果与用户交互。显示效果包括搜索结果展示、统计结果展示、关键词高亮效果等;用户交互包括查询输入框、翻页功能等。
    \item 网站控制:负责用户输入的处理与网站实际功能的实现。包括查询预处理,网站与搜索服务器的通信机制与结果处理等。
    \item 查询预处理:该模块为网站控制的一部分。负责对用户输入的预处理,过滤特殊符号并使之变为html格式的编码。
    \item 查询解析:该模块为查询平台的一部分。其目前只对英文内容进行处理,包括过滤停用词,大小写统一,过滤敏感词,单词词根化等。
    \item 数据库:索引内容的数据源,包含了全部超过1.2亿条论文的信息。所有信息被保存在7张表中,需要通过数据库的连接得到一篇论文的全部信息。
    \item 文档导入处理:该模块也是查询平台的一部分。其决定要如何从数据库中取出信息和需要取出数据库的哪些信息。
    \item 文档分析:该模块和查询解析模块的功能类似。负责对文档部分字段(例如标题字段)的处理,同样地,包括过滤停用词,大小写统一,过滤敏感词,单词词根化等步骤。
    \item 索引服务器:数据库中的数据最终被索引,以索引文件的形式保存在索引服务器中。同时,用户的查询请求也在索引服务器中被处理。同时,索引服务器还承担了关键词高亮,结果统计等扩展功能的工作。
    \end{itemize}

\section{平台配置}
    在设计完系统的整体架构后,需要对平台进行合适的配置,再能进行之后的索引结构设计与文档导入的操作。在solr平台中,主要需要配置的是solrconfig.xml文件,它管理了平台的大部分工作配置,包括启动时加载的模块,针对不同的请求的对应逻辑与系统备份、系统维护的操作等等。在一个刚刚下载好的solr服务器中,该配置文件里已经有了一些最基本的配置,我们需要通过添加一些配置以实现我们需要的功能。我们选择和本课题目标最为接近的安装包自带的data\_driven\_schema\_configs文件夹下的solrconfig.xml文件的基础上进行修改。需要进行修改的地方如下:(参考文献:solrexample)

    \begin{enumerate}
    \item 由于文档导入涉及到数据库操作,因此要在软件启动时声明加载链接mysql数据库相关的.jar包与数据导入处理相关的.jar包。在solrconfig.xml中添加以下内容:
    \begin{lstlisting}[caption={solrconfig.xml改动1}, label=ptszi1, escapeinside="", numbers=none]
        <lib dir="\$\{solr.install.dir:../../../..\}/dist/" regex="mysql-connector-java-.*\.jar" />
        <lib dir="\$\{solr.install.dir:../../../..\}/contrib/dataimporthandler/lib/" regex=".*\.jar" />
        <lib dir="\$\{solr.install.dir:../../../..\}/dist/" regex="solr-dataimporthandler-.*\.jar" />
        <lib dir="\$\{solr.install.dir:../../../..\}/contrib/dataimporthandler-extras/lib/" regex=".*\.jar" />
    \end{lstlisting}
    \item 需要配置数据导入使用的包和默认采用的配置文件。对于以HTML参数形式传入的请求,都可以在solrconfig.xml中进行处理配置。要实现此处的配置,需要在文件中添加以下内容:
    \begin{lstlisting}[caption={solrconfig.xml改动2}, label=ptszi2, escapeinside="", numbers=none]
    <requestHandler name="/dataimport" class="org.apache.solr.handler.dataimport.DataImportHandler">
    <lst name="defaults">
      <str name="config">db-data-config.xml</str>
      <str name="clean">false</str>
    </lst>
    </requestHandler>
    \end{lstlisting}
    此处意为,当HTML请求传入的第一个参数为dataimport时,使用solr中的DataImportHandler包对其进行处理。当不指定额外参数的时候,文档导入的配置文件读取文件名为db-data-config.xml的文件,且进行后一次导入时默认不清空前一次导入的索引内容。
    \item 对于大部分请求的默认域,系统默认配置为查询\_text\_字段。由于本平台对默认查询字段有特殊的要求,因此用\_entext\_字段代替\_text\_字段。此处,entext意为exglish text。关于该字段的具体说明在下一个章节中有详细解释。我们需要修改配置文件中以下对应部分的text为entext。
    \begin{lstlisting}[caption={solrconfig.xml改动3}, label=ptszi3, escapeinside="", numbers=none]
    <initParams path="/update/**,/query,/select,/tvrh,/elevate,/spell,/browse">
    <lst name="defaults">
      <str name="df">\_text\_</str>
    </lst>
    </initParams>
    \end{lstlisting}
    \item 在搜索引擎的扩展功能,即代码高亮功能与结果统计功能中,也需要对此处的配置文件进行相应的修改。在配置文件中,我们需要修改配置文件中处理select请求的部分,打开默认的高亮与统计功能的接口开关。
    \begin{lstlisting}[caption={solrconfig.xml改动4}, label=ptszi4, escapeinside="", numbers=none]
    <requestHandler name="/select" class="solr.SearchHandler">
    <lst name="defaults">
      <str name="echoParams">explicit</str>
      <int name="rows">10</int>
      <bool name="hl">true</bool>
      <str name="hl.fl">OriginalVenueName OriginalPaperTitle</str>
      <str name="hl.simple.pre">&lt;font color=#196600&gt;</str>
      <str name="hl.simple.post">&lt;/font&gt;</str>
      <bool name="facet">true</bool>
      <str name="facet.field">PaperPublishYear</str>
      <str name="facet.field">KeywordID</str>
      <str name="facet.field">AuthorID</str>
      <str name="facet.field">ConferenceSeriesID</str>
      <str name="facet.field">JournalID</str>
    </lst>
    </requestHandler>
    \end{lstlisting}
    其中hl为高亮功能,hl.simple.pre为希望设计的被高亮的字段的前缀,hl.simple.post为高亮字段后缀。由于最后高亮效果要在html页面中呈现,因此此处用了html格式的更改样式代码。facet为统计功能,facet.field声明了需要被统计的域有哪些。
    \end{enumerate}


\section{索引结构}
    本段内容描述了搜索平台中索引是如何构建的。索引的结构包括索引中需要包含哪些字段,字段是否需要被索引(index),字段是否需要被保存(store)等。被索引的字段表示我们可以通过请求查询该字段的内容,被保存的字段表示返回的结果中我们可以看到这部分的完整内容。考虑到既不需要索引也不需要保存的字段完全不需要被索引,而在本学术搜索平台的实际应用中,并不存在只需要索引且不需要保存的字段,因此,该索引中所有的字段都是需要被保存的。

    本项目中需要索引的字段共有13个。其分别为:作者ID、作者姓名、论文作者顺序、会议ID、会议简称、研究领域、期刊ID、关键词ID、关键词、论文标题、论文发表位置、论文出版年份和论文被引用数。其中,对于每一篇文章,作者ID与姓名、作者顺序、关键词ID与名称这5个字段是多值的(以列表表示),其余所有字段均是单值的;期刊ID和会议ID对于每一篇文章至多只有一项不为空;论文出版年份以数的形式表示(这代表该字段可以在搜索条件中以区间作限制),其余所有字段均以字符串的形式做表示。

    这13个字段都是被保存(STORE)的字段,它们虽然有一些并不会在网页中被直接显示,但是有可能会在统计功能的图表绘图中被使用(例如会议ID,期刊ID,关键词ID等)。同时,这13个字段中只有5个字段是被索引的,也就是论文标题,作者姓名,发表处名称,出版年份和发表会议简称。也就是用户只能通过这5个字段查询想要的内容,通过关键词搜索论文,是不能搜索到关键词被包含于这5个字段之外的字段的论文的。在这5个被索引字段中,作者姓名、出版年份和发表会议是不会被文档分析模块处理的。也就是单词词根化等操作不会影响到这些字段的内容。其余字段会被文档分析模块处理后再被建立到索引中。

    同时,这5个字段的内容在索引中都被引用到了同一个字段(\_entext\_字段)上,也就是说,平台设立了一个名叫\_extext\_的字段同时包含了这五个字段的所有内容。在默认查询中,这五个字段都会被等价地被系统检索,即系统在用户不指定的时候,会返回在\_entext\_字段的查询结果。如果用户指定了只在某个字段进行查询,例如,只在论文标题中查询输入关键字的时候,系统会返回指定字段的搜索结果。

    。。。此处添加示意图