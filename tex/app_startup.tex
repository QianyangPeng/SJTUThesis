%# -*- coding: utf-8-unix -*-
\chapter{搜索平台的启动和维护命令}

\section{单机版架构}
\subsection{启动服务}
单机版服务的启动可以直接使用example中的data import handler示例,启动方式为:
\begin{lstlisting}[basicstyle=\small\ttfamily, numbers=none]
cd solr-6.5.0
bin/solr -e dih -m 20g
\end{lstlisting}

其中-e dih指启动data import handler这一example,-m为JAVA虚拟机分配内存。启动后,基本功能都已经配置完成,接下来可以从索引结构设计开始进行进一步配置。

当然,用示例启动服务器这一行为并不优雅,我们可以用下面的指令启动一个正式的服务器:
\begin{lstlisting}[basicstyle=\small\ttfamily, numbers=none]
cd solr-6.5.0
bin/solr start -h host -p port -d dir -m memory
\end{lstlisting}

用这个指令,我们可以指定host地址,端口地址和服务器文件地址,并开启一个新的未经过配置的服务。这时我们需要从头配置所有的内容,参见第二章中平台配置小节。

\subsection{关闭服务}
关闭某个端口的服务,可以用指令:
\begin{lstlisting}[basicstyle=\small\ttfamily, numbers=none]
cd solr-6.5.0
bin/solr stop -p port
\end{lstlisting}

关闭本地所有端口的服务,可以用指令:
\begin{lstlisting}[basicstyle=\small\ttfamily, numbers=none]
cd solr-6.5.0
bin/solr stop -all
\end{lstlisting}

\section{分布式架构}
\subsection{启动服务}
分布式架构中首先需要启动Zookeeper服务:
\begin{lstlisting}[basicstyle=\small\ttfamily, numbers=none]
cd zookeeper-3.4.10
bin/zkServer.sh start
\end{lstlisting}

注意服务被启动到了哪个端口,这个端口号在conf/zoo.cfg中指定,之后需要再次用到,此处假设为1234端口。启动zkCli.sh并创建一个名为acemap的新节点:
\begin{lstlisting}[basicstyle=\small\ttfamily, numbers=none]
cd zookeeper-3.4.10
bin/zkCli.sh create nodename
\end{lstlisting}

分别在两台服务器上以云模式启动Solr服务:
\begin{lstlisting}[basicstyle=\small\ttfamily, numbers=none]
On Solr Server1:
cd solr-6.5.0
bin/solr start -h s1host -cloud -p s1port -z zkhost:1234/nodename -m 20g
On Solr Server2:
cd solr-6.5.0
bin/solr start -h s2host -cloud -p s2port -z zkhost:1234/nodename -m 20g
\end{lstlisting}

其中s1host与s2host是两台服务器的主机地址,s1port和s2port是两台服务器希望启动服务的端口号,zkhost是zookeeper服务的主机地址,1234为之前指定的端口号,nodename为之前指定的节点名称。

当服务因故退出或计算机重启后,使用以上指令可以使服务重新注册回云端。

管理云平台索引集合:
\begin{lstlisting}[basicstyle=\small\ttfamily, numbers=none]
cd solr-6.5.0
增加collection:bin/solr create -c collectionname -shards 2 -replicationFactor 2
删除collection:bin/solr delete -c collectionname
\end{lstlisting}

其中-shards表示索引分片数目,-replicationFactor表示分片备份数目,一般来说,这两个值都与分布式服务器的数目相同。

更新配置文件:
\begin{lstlisting}[basicstyle=\small\ttfamily, numbers=none]
在任何一台运行了Solr服务的计算机上:
cd solr-6.5.0/server/scripts/cloud-scripts
./zkcli.sh -z zkhost:1234/nodename -cmd upconfig -confdir configdirectory -confname configname
\end{lstlisting}

其中configdirectory为存放配置文件的本地目录,configname为节点名字(一般与collection名字相同),此命令会将配置文件传到Zookeeper的名为nodename的节点上。

更新配置文件后,需要重新加载集合:
\begin{lstlisting}[basicstyle=\small\ttfamily, numbers=none]
在任何一台运行了Solr服务的计算机上:
http://solrhost:port/solr/admin/collections?action=RELOAD&name=collectionname
\end{lstlisting}

关闭Solr服务:
\begin{lstlisting}[basicstyle=\small\ttfamily, numbers=none]
cd solr-6.5.0
bin/solr stop -p port
\end{lstlisting}

关闭Zookeeper服务:
\begin{lstlisting}[basicstyle=\small\ttfamily, numbers=none]
cd zookeeper-3.4.10
bin/zkServer.sh stop
\end{lstlisting}