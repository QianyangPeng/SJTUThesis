%# -*- coding: utf-8-unix -*-
%%==================================================
%% chapter01.tex for SJTU Master Thesis
%%==================================================

%\bibliographystyle{sjtu2}%[此处用于每章都生产参考文献]
\chapter{绪论}
\label{chap:c1}

\section{课题研究的目的和意义}

本毕设的课题为学术搜索引擎大规模查询系统的建立与优化,此课题的设立,是为了给学术网站Acemap部署一个功能完善的搜索平台。准确的说,是为该学术网站部署一个2.0版本的搜索平台。在本课题开始之前,Acemap学术网站中有一个简单的搜索平台,虽然各项功能可以正常使用,但是其索引的论文只有100万篇,仅占数据库论文总数的百分之一,且结果的展示效果也较简单粗糙,扩展功能几乎没有。究其原因,一是因为作为数据源的数据库结构复杂,建立索引时大量表连接操作的I/O密集型操作过多,导致合理时间内导入论文的数量上限很低;二是由于1.0版的搜索平台完全使用开源工具的示例代码构建,没有经过完善的设计,因此表现出功能缺乏的特征。

为了解决上述问题,部署一个2.0版本的搜索平台的需求便呼之欲出,这也是本课题设立的原因。一方面,论文需要找出一种算法,将文档的导入速度至少提升两个数量级,以完成全部超过一亿篇论文的索引建立。如果完整的索引得以建立完成,由于我们对如此大规模的查询系统在服务器上运行的经验几乎为零,因此届时系统中可能会出现各种各样的问题,也需要我们去进一步解决。另一方面,论文需要抛弃使用示例代码构建的查询系统,而需要从头开始配置与部署一个符合需求的搜索平台,并实现搜索平台的分布式部署。

为了增加课题理论方面的意义,也为了更紧密地结合我的科研小组的研究重心,我在本次毕设中还计划开发一个基于查询系统的知识图功能。本功能的目标是在用户输入关键词进行查询的时候,能自动基于查询结果的关键词等信息生成一张知识脉络图。例如对于energy saving这一输入,系统能在环境科学、经济学、电气、化学、数学等领域梳理出相关研究的层级状知识体系图,以帮助用户更好地确定自己的研究方向。

\section{课题的理论基础}

与查询系统的1.0版本一样,我们使用开源搜索服务器工具Solr\citen{apachesolr}作为系统搭建的平台。Solr基于著名的全文检索引擎架构Lucene\citen{lucene}开发,其本质是使用倒排索引技术实现的全文检索工具。倒排索引的“倒排”二字,即为将原数据的文档ID对应关键词的对应关系倒转为关键词对应文档ID列表的过程。经过此步处理后,根据关键词或关键词组合查找文档的过程即可以用传统的索引方式完成。一般要增加搜索引擎的可用性,我们需要用一些自然语言处理的技术对文档和查询语句进行处理,例如去除停用词(filter stopwords)和词干提取(stemming)\citen{stemming}等。

在分布式系统中,我们使用开源工具Zookeeper\citen{zookeeper}作为分布式程序协调服务平台。它在提供了集中的配置文件托管平台的同时,也解决了分布式应用中的死锁和数据同步的问题。死锁问题可能使多个节点同时提交请求修改同一内容时,导致所有的请求都不能成功提交。Zookeeper采用选举领导(Leader)的方式解决了这一问题,在Zookeeper管理的节点集合中,只有领导节点可以进行请求的提交操作。Zookeeper的内部采用四种算法进行领导节点的选举,但是一般采用依赖逻辑时钟的FastLeaderElection的算法。算法的思想基于Leslie Lamport的论文\citen{leslie},采用竞选请求加入队列,节点投票式选举的方法选出领导节点。

在知识图部分,很多论文对我的想法的构思起到了很大的启发作用,也有论文针对关键词的节点聚合与主题提取算法为我的算法设计提供了思路。针对这一部分,在论文的第六章中会有详细的说明。

\section{论文结构}
本课题的工作主要分为两个部分,一是查询系统主体的构建,二是知识图功能的相关研究。查询系统主体部分的工作可以列举如下:

\begin{enumerate}
  \item 系统架构与索引结构的设计
  \item 大规模数据的高效导入
  \item 搜索后台与前端的接口设计
  \item 查询系统的分布式部署
\end{enumerate}

本论文在第二到五章依次讨论了这些问题。事实上,本文也是一个偏设计类的论文。在研究的过程中,本人阅读了大量的官方与非官方的著作与相关文档,但由于学术搜索这一应用较为特殊,数据库中总条目数很多且数据结构复杂,这导致参考文档只起到了极为有限的参考作用。在大部分的成果中,本人的工作内容都是自己设计,实现并测试优化的。因此,这项建立学术搜索平台的工作,不仅在设计上具有原创性和完整性,也针对遇到的棘手问题有独到的解决方案,这也坚定了我将自己的工作整理写作成论文的信念。第二部分中,论文讨论了如何在搜索结果的基础上建立层级化知识图,这一部分的工作在第六章中有详细的说明。