%# -*- coding: utf-8-unix -*-
%%==================================================
%% chapter01.tex for SJTU Master Thesis
%%==================================================

%\bibliographystyle{sjtu2}%[此处用于每章都生产参考文献]
\chapter{绪论}
\label{chap:c1}

本毕业论文的题目为学术搜索引擎大规模查询系统的建立与优化,论文主要讨论如何为学术搜索网站建立一个高效稳定的大规模查询系统。在此基础上,论文还讨论了如何对系统进行优化,功能扩展与分布式部署,同时研究了如何设计根据搜索条件动态生成知识层级图的算法。

论文中所有的实验,系统部署与验证都在学术搜索平台ACEMAP上进行。目前绝大多数学术网站也都提供了功能完善的论文搜索功能,因而在acemap中部署搜索平台的工作也是必要的。如何设计并实现一个大规模的查询系统是一个较为复杂的工程问题,在技术细节上也会有很多难点与挑战。在这一部分的工作中,主要的工作可以列举如下:

\begin{enumerate}
  \item 系统架构与索引结构的设计
  \item 大规模数据的高效导入
  \item 搜索后台与前端的接口设计
  \item 查询系统的分布式部署
\end{enumerate}

本论文在第二到五章依次讨论了这些问题,这构成了本篇论文的主体部分。事实上,本文也是一个偏设计类的论文。在研究的过程中,本人阅读了大量的官方与非官方的著作与相关文档,但由于学术搜索这一应用较为特殊,数据库中总条目数很多且数据结构复杂,这导致参考文档只起到了极为有限的参考作用。在大部分的成果中,本人的工作内容都是自己设计,实现并测试优化的。因此,这项建立学术搜索平台的工作,不仅在设计上具有原创性和完整性,也针对遇到的棘手问题有独到的解决方案,这也坚定了我将自己的工作整理写作成论文的信念。

在第六章中,论文讨论了如何在搜索结果的基础上建立层级化知识图。该功能的构想基于如下考量:用户使用学术网站进行查询的时候,需要的除本领域的相关知识,还可能需要一些跨领域的知识。在用户输入关键词进行查询的时候,可能只重点关注搜索结果的前几条,而隐藏在大量搜索结果之中的知识信息则无法很好的被用户知晓。论文希望给出一种算法,可以基于查询结果自动生成知识脉络图,挖掘出和用户搜索关键词相关的知识信息。例如对于energy saving这一输入,系统能在环境科学、经济学、电气、化学、数学等领域梳理出相关研究的层级状知识图,以帮助用户更好地确定与了解自己的研究方向。

