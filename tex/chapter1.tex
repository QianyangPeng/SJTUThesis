%# -*- coding: utf-8-unix -*-
%%==================================================
%% chapter01.tex for SJTU Master Thesis
%%==================================================

%\bibliographystyle{sjtu2}%[此处用于每章都生产参考文献]
\chapter{绪论}
\label{chap:c1}

本毕业设计的题目为学术搜索引擎大规模查询系统的建立与优化,其目的是为学术搜索网站ACEMAP(网址)建立一个高效稳定的查询系统,并在网站中实现完整的搜索功能。在此基础上,论文还讨论了如何对系统进行优化,功能扩展与分布式部署,同时研究了如何设计根据搜索条件动态生成知识层级图的算法。

ACEMAP目前以学术地图作为主打功能,学术地图能够让用户方便地对研究领域有一个直观的认识,但是如果用户需要在总共约1.2亿篇论文中方便且精确地定位到某一篇论文,一个成熟完整的查询系统是必不可少的。目前绝大多数学术网站,也都提供了功能完善的论文搜索功能。如何设计并实现一个较大规模的查询系统是一个较为复杂的工程问题,在技术细节上也会有很多难点与挑战。在这一部分的工作中,主要的工作可以列举如下:

\begin{enumerate}
  \item 系统架构与索引结构的设计
  \item 大规模数据的高效导入
  \item 系统优化与功能扩展
  \item 系统分布式部署与备份
\end{enumerate}

本论文在第二到五章依次讨论了这些问题。而在第六章中,论文讨论了如何在搜索结果的基础上建立层级化知识图。该功能的构想基于如下考量:用户使用学术网站进行查询的时候,需要的除本领域的相关知识,还可能需要一些跨领域的知识。在用户输入关键词进行查询的时候,可能只重点关注搜索结果的前几条,隐藏在成百上千条论文结果之中的知识信息则无法很好的被用户知晓。论文希望给出一种算法,可以基于查询结果自动生成知识脉络图,挖掘出和用户搜索关键词相关的知识信息。例如对于energy saving这一输入,系统能在环境科学、经济学、电气、化学、数学等领域梳理出相关研究的层级状知识图,以帮助用户更好地确定与了解自己的研究方向。