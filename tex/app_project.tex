%# -*- coding: utf-8-unix -*-
\chapter{索引文件映射到内存的方法}

\section{定位索引文件位置}
首先我们需要知道索引文件存放的位置。一般来说,Solr的索引文件存放于文件目录的./data/index目录下
\begin{lstlisting}[basicstyle=\small\ttfamily, numbers=none]
Index: /home/path/to/solr/collection/data/index
\end{lstlisting}

\section{备份索引文件夹}
\begin{lstlisting}[basicstyle=\small\ttfamily, numbers=none]
cp ./index ./index2
\end{lstlisting}

\section{将索引文件夹挂载到内存上}
\begin{lstlisting}[basicstyle=\small\ttfamily, numbers=none]
mount -t tmpfs -o size=90g tmpfs ./index
\end{lstlisting}
该操作会花费90G内存空间,如果计算机内存空间不足,可酌情减少。不过,该值太小会导致solr服务无法启动。

\section{将索引文件拷贝回目标目录}
\begin{lstlisting}[basicstyle=\small\ttfamily, numbers=none]
cp -Rf ./index2/* ./index
\end{lstlisting}
注意不要用mv指令直接剪切并粘贴文件。因为内存中保存的文件不稳定,所以请保留一份硬盘备份。

\section{按正常方式启动Solr服务}
\begin{lstlisting}[basicstyle=\small\ttfamily, numbers=none]
./bin/solr start
\end{lstlisting}